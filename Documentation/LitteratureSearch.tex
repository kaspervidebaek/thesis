
\title{Litterature search}
\author{
        Kasper Videbæk \\
}
\date{\today}

\documentclass[12pt]{article}
\usepackage[hidelinks]{hyperref}

\begin{document}
\maketitle

\begin{abstract}
This documents my literature search method and result.
\end{abstract}

\section{Problem formulation}
The overall goal of the project is to minimize conflicts happening in version control systems, when merging source code, by merging on the syntax tree level. This creates a few different problem formulations which might be present in the current literature. Below is a list of three questions that seem relevant.

All of the below search words, and combinations thereof have been included in a search on Google Scholar.

\subsection{What work have been on merging syntax tree in VCSs so far?}
This problem formulation is the most general, straightforward to derive out of the overall problem formulation, and no explanation seems necessary. The term-map of the search can be found below.

\begin{tabular}{ | l | l | l | l |}
    \hline
    \textbf{Term} & Version Control & Merge & Syntax Tree \\ \hline
    \textbf{Major term} & Changes & & Fine grained, Structure" \\ \hline
    \textbf{Minor term} & git, svn, hg, cvs  & diff, patch & \\ \hline
\end{tabular}

\subsection{What work have been on creating generalized syntax trees so far?}
The idea here is to research whether there exists literature that has tried to generalize syntax trees between languages which would hopefully help when we need to do the same.

\begin{tabular}{ | l | l | l | l | }
    \hline
    \textbf{Term} & Syntax Tree & Generalization \\ \hline
    \textbf{Major term} & Structure & Abstraction \\ \hline
\end{tabular}


\subsection{What work have been done on general tree change detection?}
Tree change detection is quite an algorithmic problem. The point here is to find work on this specific topic and identify techniques to minimize the running time of a tree differencing algorithm.

\begin{tabular}{ | l | l | l | l |}
    \hline
    \textbf{Term} & Tree & diff, difference \\ \hline
    \textbf{Major term} & Structure, Graph & Change detection \\ \hline
    \textbf{Minor term} & XML & & Edit distance
\end{tabular}

\clearpage
\subsection{Results}
VCS merging: \\
\begin{tabular}{ | l | l | }
	1989 & \href{http://dl.acm.org/citation.cfm?id=65980}{Integrating noninterfering versions of programs } \\
	1991 & \href{http://dl.acm.org/citation.cfm?id=111071}{Structure-oriented merging of revisions of software documents } \\
	1992 & \href{http://dl.acm.org/citation.cfm?id=143753}{Operation-based merging} \\
	1995 & \href{http://ieeexplore.ieee.org/xpls/abs_all.jsp?arnumber=393613}{Version management for tightly integrated software engineering environments } \\
	1996 & \href{http://link.springer.com/chapter/10.1007%2FBFb0023079?LI=true0}{(Kasper Østerby PDF) Fine Grained Version Control of Configurations in COOP/Orm } \\
	2002 & \href{http://citeseerx.ist.psu.edu/viewdoc/summary?doi=10.1.1.228.5188}{Extensible language-aware merging} \\
	2002 & \href{http://ieeexplore.ieee.org/xpls/abs_all.jsp?arnumber=1000449}{A state-of-the-art survey on software merging } \\
	2004 & \href{http://www.sciencedirect.com/science/article/pii/S1571066104051916}{Refactoring-aware versioning in Eclipse } \\
	2006 & \href{http://dl.acm.org/citation.cfm?id=1134461}{Refactoring-aware version control } \\
	2011 & \href{http://scholar.google.dk/scholar?cluster=3247494458869222515&hl=en&as_sdt=2005&sciodt=0,5}{Rethinking Merge in Revision Control Systems} \\
	2012 & \href{http://www.infosun.fim.uni-passau.de/spl/JDime/deploy/ma_olaf.pdf}{Adjustable Syntactic Merge of Java Programs } \\
\end{tabular}
\\
VCS diffing\\
\begin{tabular}{ | l | l | }
	1991 & \href{http://citeseerx.ist.psu.edu/viewdoc/summary?doi=10.1.1.13.9377}{Identifying Syntactic Differences Between Two Programs} \\
	
	1998 & \href{http://128.148.32.110/research/pubs/pdfs/1998/Klein-1998-CED.pdf}{Computing the Edit-Distance Between Unrooted Ordered Trees} \\
	2002 & \href{http://ieeexplore.ieee.org/xpls/abs_all.jsp?arnumber=1167813}{A fine-grained version and configuration model in analysis and design } \\
	2005 & \href{http://ieeexplore.ieee.org/xpl/articleDetails.jsp?tp=&arnumber=1532125}{The visual code navigator: an interactive toolset for source code investigation	} \\
	2006 & \href{http://dl.acm.org/citation.cfm?id=1137999}{Program element matching for multi-version program analyses. } \\
	2006 & \href{http://ieeexplore.ieee.org/xpls/abs_all.jsp?arnumber=1631103&tag=1}{Classifying Change Types for Qualifying Change Couplings } \\
	2007 & \href{http://ieeexplore.ieee.org/xpl/articleDetails.jsp?tp=&arnumber=4339230}{Change Distilling:Tree Differencing for Fine-Grained Source Code Change Extraction } \\
	2008 & \href{http://ieeexplore.ieee.org/xpl/articleDetails.jsp?tp=&arnumber=4656419}{Diff/TS: A Tool for Fine-Grained Structural Change Analysis } \\
\end{tabular}
\\
VCS Archeology\\
\begin{tabular}{ | l | l | }
	2009 & \href{http://ieeexplore.ieee.org/xpl/articleDetails.jsp?tp=&arnumber=4721180}{Change Analysis with Evolizer and ChangeDistiller } \\
\end{tabular}
\\
Including semantics: \\
\begin{tabular}{ | l | l | }
	2004 & \href{http://ieeexplore.ieee.org/xpls/abs_all.jsp?arnumber=1357803&tag=1}{Dex: a semantic-graph differencing tool for studying changes in large code bases } \\
	2009 & \href{http://www.andres-loeh.de/gdiff-wgp.pdf}{Type-Safe Diff for Families of Datatypes } \\
\end{tabular}
\\
Unstructured merge: \\
\begin{tabular}{ | l | l | }
	2007 & \href{http://www.cis.upenn.edu/~bcpierce/papers/diff3-short.pdf}{A Formal Investigation of Diff3 } \\
	2009 & \href{http://ieeexplore.ieee.org/xpl/articleDetails.jsp?tp=&arnumber=5070564}{Ldiff: An enhanced line differencing tool } \\
\end{tabular}
\\
Higer-order abstract syntax: \\
\begin{tabular}{ | l | l | }
	1988 & \href{http://dl.acm.org/citation.cfm?id=54010}{Higher-order abstract syntax} \\
	1995 & \href{http://ieeexplore.ieee.org/xpls/abs_all.jsp?arnumber=514712&tag=1}{Reengineering procedural into object-oriented systems} \\	
\end{tabular}
\\
Tree diffing algorithms: \\
\begin{tabular}{ | l | l | }
	1995 & \href{http://citeseerx.ist.psu.edu/viewdoc/summary?doi=10.1.1.48.8675}{A Parallel Tree Difference Algorithm } \\
	1995 & \href{http://ilpubs.stanford.edu:8090/115/1/1995-46.pdf}{Change Detection in Hierarchically Structured Information} \\
	2001 & \href{http://tdm.berlios.de/3dm/doc/thesis.pdf}{A 3-way Merging Algorithm for Synchronizing Ordered Trees} \\
	2003 & \href{http://www.lifl.fr/~touzet/Publications/cpm03.pdf}{Analysis of Tree Edit Distance Algorithms} \\
	2003 & \href{http://www.sciencedirect.com/science/article/pii/S0167865502002556}{Computing approximate tree edit distance using relaxation labeling} \\
	2003 & \href{http://www.sciencedirect.com/science/article/pii/0304397595800299}{Alignment of trees — an alternative to tree edit} \\
	2003 & \href{http://ieeexplore.ieee.org/xpls/abs_all.jsp?arnumber=1260818}{X-Diff: An Effective Change Detection Algorithm for XML Documents} \\
	2005 & \href{http://citeseerx.ist.psu.edu/viewdoc/summary?doi=10.1.1.100.2577}{A survey on tree edit distance and related problems } \\
	2007 & \href{http://link.springer.com/chapter/10.1007\%2F978-3-540-73420-8_15?LI=true}{An Optimal Decomposition Algorithm for Tree Edit Distance}  \\ % http://stackoverflow.com/questions/1065247/how-do-i-calculate-tree-edit-distance
	2011 & \href{http://arxiv.org/abs/1201.0230}{RTED: A Robust Algorithm for the Tree Edit Distance}
\end{tabular}
\\
Other: \\
\begin{tabular}{ | l | l | }
	2004 & \href{http://ieeexplore.ieee.org/xpls/abs_all.jsp?arnumber=1357805}{Supporting source code difference analysis } \\
\end{tabular}

Three-way merge:
http://tdm.berlios.de/3dm/doc/thesis.pdf
Identifying Conflicts During Structural Merge 	http://citeseerx.ist.psu.edu/viewdoc/summary?doi=10.1.1.48.3795
Integrating non-interfering versions of programs (1989)  http://citeseerx.ist.psu.edu/viewdoc/summary?doi=10.1.1.115.298



\section{Metode i dette dokument}
Først skriver jeg et regerat for hver enkelt artikel. Herefter finder jeg nogle temaer og grupperer hver enkelt af dem.

\section{Introduction}
We are going to draw from knowledge in several fields to find the correct way of merging source code through syntax trees. We will look into general approaches for differencing trees, and for merging them. Further we are going to look 	

\subsection{Operation-based Merging}
Describes that merging tools usually work in a state-based manner, and suggest that for some document types, a operation based approach might be better. These operations should be logged during changing documents, and applied by the diffing. Spends alot of time on conflict resolving, that will naturally occur on such opreations, and describes how to solve these. This relates much to many of the atomic operations we want to perform on syntax trees.

\subsection{Diff/Ts: A Tool for Fine-Grained Structural Change Analysis}
Describes a implementation of a tool to differencing structured changes in programs in VCS - have implementation for 4 different Object-oriented languages. It also discusses several applications. Describes that too fine-grained Diffs will actually be less comprehensible than changing the entire node. Describes a differencing algorithm as something that matches trees and afterwords creates an edit-script.

\subsection{Refactoring aware version control}
Describes the conceptual idea of embedding refectoring aware changes inside version control systems. The idea is not to merge, but to let version control systems do automatic refactoring when new files are checked in that uses an outdated API. 

\subsection{Change distilling: Tree Differencing for Finegrained source code change extraction}
Describes a diffing algorithm for trees, with specific details about code. It has the following edit operations: Instert, Delete, Move, Update(Value) and Allignemnt. Describes how dynamic heuristics can be used to define how gooda match a node is to another node (in both small and big cases of blocks). Also describes string similiarity measures which would be good for reordering of words in variablenames. Is an approximation algorithm.

\subsection{A Survey on Tree Edit Distances.}
Compares and provides a common framework for many tree differencing algorithms. Provides a nice table with overview of running time for different algorithms. Also describes the idea of constraining the edit distance we are interested in, to allow for better running times.

\subsection{Extensible Language-Aware merging}
Wants to merge on the semantic level, however only in very simplistic ways (e.g. test for side-effects of variables) - use heuristics. Propogates name changes between edit scripts before applying them to the base.

\subsection{Refactoring-aware versioning in Eclips}
Merges with "Rename" and "Move" added. Many of the same things described as my thoughts. Does not talk about running times.

\subsection{Adjustable Syntactic Merge of Java Programs}
Thesis that does textual merge and starts a syntactic merge when this fails. Does not understand renames and only moves in some areas. Seems to use exact algorithms. Uses 3-way merging.

\subsection{A Formal Investigation of Diff3}
Investigates foramlly how Diff3 works. Defines a couple of properties that intuitively should be true for merging:
- Locality of changes should not produce conflicts. (Not true unless unqieie elements)
- Idempotence: When a diff has been applied, applying the same diffs to the result should yield no change.
- Diff3 fails for very close versions of A and B, when they are very far from base.
- Stability in the closeness on similir inputs (but different orders)


\bibliographystyle{abbrv}
\bibliography{main}

\end{document}
